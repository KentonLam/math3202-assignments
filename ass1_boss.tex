\documentclass[11pt,a4paper]{article}
\input{../../../header.tex}
\input{../../../normal.tex}

\usepackage{lmodern}
\renewcommand*\familydefault{\sfdefault} %% Only if the base font of the document is to be sans serif
\usepackage[T1]{fontenc}

\author{Kenton Lam}
\date{{MATH3202 Assignment 1 \\ Due 15/03/2019 12:00 pm}}
\title{WonderMarket Internal Report}


\begin{document}
\maketitle
\begin{abstract}
    We have been approached by WonderMarket to optimise their supply chain logistics.
    Given information about their costs and needs, we created and optimmised a model using Gurobi.
    This report describes the model and its solution.
\end{abstract}

\part{Model Definition}
WonderMarket operates a number of stores and distribution centres (DCs).
Each week, each store requires some amount of goods (measured in truckloads)
to be delivered from one or more of their distribution centres (DC).

\section{Sets}
\begin{tabular}{l l}
    $S$ & set of stores. \\ 
    $D$ & set of distribution centres.    
\end{tabular}

\section{Data}
\begin{tabular}{l l}
    $C_{d,s}$ & cost of moving one truckload of goods from DC $d$ to store $s$ (comm 1). \\ 
    $R_s$ & number of truckloads required at store $s$ per week (comm 1). \\
    $M_d$ & maximum capacity (truckloads) which can be send from $d$ per week (comm 2).
\end{tabular}

\section{Variables}
\begin{tabular}{l l}
    $X_{d,s}$ & number of truckloads to be moved from DC $d$ to store $s$.
\end{tabular}

\section{Objective}
\begin{align*}
    \min \sum_{s \in S} \sum_{d \in D} C_{d, s} X_{d,s}
\end{align*}

\section{Constraints}
\subsection{Communication 1}
Each store receives sufficient goods. For $s \in S$,
\begin{align*}
    \sum_{d \in D} X_{d,s} \ge R_s.
\end{align*}

\subsection{Communication 2}
Each distribution centre's capacity is not exceeded. For $d \in D$,
\begin{align*}
    \sum_{s \in S} X_{d,s} \le M_d.
\end{align*}



\end{document}