\documentclass[11pt,a4paper]{article}
\input{../../../header.tex}
\input{../../../normal.tex}

\usepackage{lmodern}
\renewcommand*\familydefault{\sfdefault} %% Only if the base font of the document is to be sans serif
\usepackage[T1]{fontenc}

\author{Kenton Lam}
\date{{MATH3202 Assignment 1 \\ Due 29/03/2019 12:00 pm}}
\title{WonderMarket \\ 
Section B -- Client Report}


\begin{document}
\maketitle
\begin{abstract}
    In this report, we propose a solution to optimise your supply chain logistics, 
    to minimise costs while ensuring all demands are met.
\end{abstract}

\part{Solution}
We have considered your requirements of weekly demand, DC capacity, northside capacity and surge demands.
We propose the following assignment of stores to distribution centres for each communication.

\section{Communication 1}
Only considering each store's demand without any capacity constraints 
resulted in a cost of \$150212. The solution is described below (in truckloads per week). \\[0.8em]
\begin{tabular}{l  r  r  r }
    Store & DC0 & DC1 & DC2 \\
    S0 & 0.00 & 0.00 & 18.00 \\
    S1 & 7.00 & 0.00 & 0.00 \\
    S2 & 0.00 & 0.00 & 21.00 \\
    S3 & 0.00 & 0.00 & 15.00 \\
    S4 & 0.00 & 0.00 & 17.00 \\
    S5 & 0.00 & 0.00 & 10.00 \\
    S6 & 0.00 & 6.00 & 0.00 \\
    S7 & 0.00 & 0.00 & 8.00 \\
    S8 & 0.00 & 0.00 & 7.00 \\
    S9 & 7.00 & 0.00 & 0.00 \\    
\end{tabular}
\pagebreak

\section{Communication 2}
With DC capacity considered, the cost was  \$174952 with 
the following assignments (truckloads per week). \\[0.8em]
\begin{tabular}{l  r  r  r }
    Store & DC0 & DC1 & DC2 \\
S0 & 3.00 & 0.00 & 15.00 \\
S1 & 7.00 & 0.00 & 0.00 \\
S2 & 21.00 & 0.00 & 0.00 \\
S3 & 0.00 & 15.00 & 0.00 \\
S4 & 17.00 & 0.00 & 0.00 \\
S5 & 0.00 & 0.00 & 10.00 \\
S6 & 0.00 & 6.00 & 0.00 \\
S7 & 0.00 & 0.00 & 8.00 \\
S8 & 0.00 & 0.00 & 7.00 \\
S9 & 7.00 & 0.00 & 0.00 \\
\end{tabular}

\section{Communication 3}
With the northside capacity limit as well, the cost was \$179882 with the following 
assignments (truckloads per week).\\[0.8em]
\begin{tabular}{l r r r}
    Store & DC0 & DC1 & DC2 \\
S0 & 0.00 & 0.00 & 18.00 \\
S1 & 7.00 & 0.00 & 0.00 \\
S2 & 21.00 & 0.00 & 0.00 \\
S3 & 0.00 & 15.00 & 0.00 \\
S4 & 17.00 & 0.00 & 0.00 \\
S5 & 0.00 & 3.00 & 7.00 \\
S6 & 0.00 & 6.00 & 0.00 \\
S7 & 0.00 & 0.00 & 8.00 \\
S8 & 0.00 & 0.00 & 7.00 \\
S9 & 0.00 & 7.00 & 0.00 \\
\end{tabular}
\pagebreak

\section{Communication 4}
Considering the surge scenarios, we developed the following assignment, with a 
regular (non-surge) cost of \$199661.44 per week.
We have verified that these assignments can scale up to the surge scenarios provided
without exceeding any of your capacity limits.
The following are the percentages of demand each store should receive 
from each distribution centre. \\[0.8em]
\begin{tabular}{l r r r}
    Store & DC0 & DC1 & DC2 \\
S0 & 0.00\% & 0.00\% & 100.00\% \\
S1 & 100.00\% & 0.00\% & 0.00\% \\
S2 & 100.00\% & 0.00\% & 0.00\% \\
S3 & 0.00\% & 100.00\% & 0.00\% \\
S4 & 58.82\% & 0.00\% & 41.18\% \\
S5 & 0.00\% & 83.86\% & 16.14\% \\
S6 & 0.00\% & 100.00\% & 0.00\% \\
S7 & 0.00\% & 64.77\% & 35.23\% \\
S8 & 0.00\% & 64.77\% & 35.23\% \\
S9 & 100.00\% & 0.00\% & 0.00\% \\
\end{tabular}\\
Applying these percentages to normal demand, we have the following truckloads per 
week. \\[0.8em]
\begin{tabular}{l r r r }
    Store & DC0 & DC1 & DC2 \\
S0 & 0.00 & 0.00 & 18.00 \\
S1 & 7.00 & 0.00 & 0.00 \\
S2 & 21.00 & 0.00 & 0.00 \\
S3 & 0.00 & 15.00 & 0.00 \\
S4 & 10.00 & 0.00 & 7.00 \\
S5 & 0.00 & 8.39 & 1.61 \\
S6 & 0.00 & 6.00 & 0.00 \\
S7 & 0.00 & 5.18 & 2.82 \\
S8 & 0.00 & 4.53 & 2.47 \\
S9 & 7.00 & 0.00 & 0.00 \\
\end{tabular}

\part{Insights}
\section{General Observations}
While optimising the assignments with the data you provided us, we made 
the following observations.
\begin{itemize}
    \item With the exception of surge scenario 4, the capacity at DC1 is never 
fully used. Normal demand and scenarios 0 -- 3 use at most 54 truckloads per week 
from DC1. However, reducing its throughput (by any amount) would make surge scenario 4
impossible.
    \item DC0's capacity of 72 is never fully utilised. At most, 45.6 truckloads are 
    required from it per week (during scenario 4).
    \item Surge scenario 4 is the limit of your current system. It requires 161 truckloads 
    in total. Because DC0 and DC2 together can only supply 85 and DC1 can supply 76, this is 
    exactly 161 truckloads per week. 
\end{itemize}

\section{Marginal Costs}
These are the effects of slight adjustments to the capacity of DCs 
 you have provided us. These may be useful to improve your management of 
 distribution centres. 
\begin{itemize}
    \item Increasing DC0 or DC1's capacity 
    (by any amount) will not reduce the cost, as they are not bottlenecks of your supply chain.
    \item Increasing DC2's capacity by 1 truckload would 
    reduce the regular demand cost by \$496, to \$199165.44.
    \item Increasing the northside capacity by 1 truckload per week would reduce your
    normal demand cost by \$387.02, to \$199274.43. The northside is a limiting 
    constraint in 3 of the 5 surge scenarios.
\end{itemize}

\end{document}