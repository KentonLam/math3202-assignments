\documentclass[11pt,a4paper]{article}
\input{../../../../header.tex}
\input{../../../../normal.tex}

\usepackage{booktabs}

\usepackage{lmodern}
\renewcommand*\familydefault{\sfdefault} %% Only if the base font of the document is to be sans serif
\usepackage[T1]{fontenc}

\author{Kenton Lam}
\date{{MATH3202 Assignment 2 \\ Due 03/05/2019 1:00 pm}}
\title{WonderMarket \\ Section A -- Internal Report}


\begin{document}
\maketitle
\begin{abstract}
    We have been approached by WonderMarket to optimise their supply chain logistics.
    Given information about their costs and needs, we created and optimmised a model using Gurobi.
    This report describes the model and its solution.
\end{abstract}

\part{Model Definition}
WonderMarket operates a number of stores and distribution centres (DCs).
Each week, each store requires some amount of goods (measured in truckloads)
to be delivered from one or more of their distribution centres (DC).

After communication 9, we developed a detailed model to mathematically represent
their objectives. This includes handling surge scenarios and predicting the 
total yearly cost of their operations.

\section{Sets}
\begin{tabular}{l l}
    $S$ & set of stores. \\ 
    $D$ & set of distribution centres (this includes proposed new DCs). \\ 
    $U$ & set of surge scenarios.   
\end{tabular}

\section{Data}
\begin{tabular}{l l}
    $\mathrm{Cost}_{d,s}$ & cost of moving one truckload of goods from DC $d$ to store $s$ (comm 1). \\
    &(this includes cost of transport from new DCs after communication 6).\\ 
    $\mathrm{Demand}_s$ & number of truckloads required at store $s$ per week normally (comm 1). \\
    $\mathrm{Capacity}_d$ & maximum capacity (truckloads) which can be sent from $d$ per week (comm 2).  \\
    & (this includes capacities of new DCs after communication 6).\\ 
    $\mathrm{Northside}$ & set of distribution centres on the north side (comm 3). \\ 
    $\mathrm{NorthsideMax}$ & maximum combined capacity of northside DCs (comm 3). \\ 
    $\mathrm{SurgeDemand}_{u,s}$ & surge demand during scenario $u$ at store $s$ (comm 4). \\ 
    $\mathrm{NewDCs}$ & set of DCs which are not currently built (comm 6).\\ 
    $\mathrm{SurgeWeeks}_u$ & number of weeks surge scenario $u$ occurs per year (comm 9). 
\end{tabular}

\subsection{Derived Data}
To simplify later calculations, we derived some data from the data given above.\\[0.8em]    
\begin{tabular}{l l}
    $\mathrm{SurgeMultiplier}_{u,s}$ & surge demand at store $s$ during surge $u$,  divided by regular \\ 
    &  demand at store $s$. \\ 
    &$\mathrm{SurgeMultiplier}_{u,s} = \mathrm{SurgeDemand}_{u,s} / \mathrm{Demand}_s $ \\
    $\mathrm{NormalWeeks}$ & number of weeks in the year which have no surge scenario. \\ 
    & $\mathrm{NormalWeeks} = 52-\sum_{u \in U} \mathrm{SurgeWeeks}_u $
\end{tabular}

\section{Variables}
The following variables were used in the Gurobi model.\\[0.8em]
\begin{tabular}{l l}
    $B_{d}$ & binary variables for whether DC $d$ is active. \\ 
    & (1 means if $d$ is new then $d$ is built; if $d$ 
    already exists then $d$ is not closed). \\
    $P_d$ & integer number of part-time teams employed at DC $d$ year-round. \\ 
    $F_d$ & integer number of full-time teams employed at DC $d$ year-round. \\ 
    $C_{u,d}$ & integer number of casual employeed employed at DC $d$  during surge $u$. \\
    & (each casual employee is only employed for the duration of the surge). \\ 
    $A_{d,s}$ & binary variable for whether DC $d$ delivers to store $s$. \\  
    $X_{d,s}$ & truckloads to be sent from DC $d$ to store $s$ during normal demand. \\
    $Y_{d,s,u}$ & truckloads to be sent from DC $d$ to store $s$ during 
    surge scenario $u$. 
\end{tabular}
\\[0.8em]

\section{Objective}
This calculates the total yearly cost, considering transport and labour costs.
The objective is to \textit{minimise} the following function.
\begin{align*}
    \text{(normal transport cost)} \qquad&\quad \sum_{s \in S} \sum_{d \in D} \mathrm{NormalWeeks} \cdot \mathrm{Cost}_{d, s}  X_{d,s} \\ 
    \text{(surge transport costs)} \qquad &+ \sum_{u \in U} \sum_{s \in S} \sum_{d \in D} \mathrm{SurgeWeeks}_u \mathrm{Cost}_{d, s} Y_{d,s,u} \\ 
    \text{(full/part-time labour costs)} \qquad &+ 52 \sum_{d \in D} 2750\, P_d + 52 \sum_{d \in D}4500 \,F_d \\
    \text{(casual labour costs)} \qquad &+\sum_{u \in U} \sum_{d \in D} 2951\,\mathrm{SurgeWeeks}_u C_{u,d}
\end{align*}

\section{Constraints}
First, we have the constraint that all variables are non-negative and certain variables 
are integers or binary. For all $d \in D,~s \in S, ~u \in U$, we have
\begin{align*}
    B_d, P_d, F_d, C_{u,d}, A_{d,s}, X_{d,s}, Y_{d,s,u}  &\ge 0  \\ 
    B_d, A_{d,s} &\in  \left\{ 0, 1\right\} \\ 
    P_d, F_d, C_{u,d} &\in \mathbb Z
\end{align*}
We link $X$ and $Y$ variables via $A$ and the known demand at each store. This ensures 
each store receives all its supplies from one DC.
\begin{align*}
    X_{d,s} &= \mathrm{Demand}_s A_{d,s} && \forall~d \in D, ~s \in S
\end{align*}
We link $X$ and $Y$ via the surge multipliers data. This ensures for each store, 
it receives deliveries from the same DCs in each scenario as during normal demand. 
\begin{align*}
    Y_{d,s,u} &= \mathrm{SurgeMultiplier}_{u,s} X_{d,s} && \forall~d \in D,~s \in S,~u \in U
\end{align*}
We ensure the solution is valid for normal demand considering 
truckloads and capacities. Note that the northside capacity limit has been removed.
\begin{align*}
    \sum_{d \in D} X_{d,s} &\ge \mathrm{Demand}_{s} &&\forall~s \in S \\ 
    \sum_{s \in S} X_{d,s} &\le \mathrm{Capacity}_d &&\forall~d \in D 
\end{align*}
Ensuring at most 2 new DCs are built and there are 4 DCs in total (3 or 2 old).
\begin{align*}
    \sum_{d \in \mathrm{NewDCs}} B_d &\le 2 \\ 
    \sum_{d \in D} B_d &= 4 
\end{align*}
Ensuring enough teams are employed to meet normal demand at each DC.
\begin{align*}
    \sum_{s \in S} X_{d,s} &\le 9\,F_d + 5\,P_d && \forall~d \in D
\end{align*}
We ensure our assignments can scale up to each surge scenario while remaining 
feasible with the given constraints. 
For each $u \in U$,
\begin{align*}
    \sum_{d \in D} Y_{d,s,u} &\ge \mathrm{SurgeDemand}_{u,s} &&\forall~s \in S \\ 
    \sum_{s \in S} Y_{d,s,u} &\le \mathrm{Capacity}_d &&\forall~d \in D \\ 
    \sum_{s \in S} Y_{d,s,u} &\le 9\,F_d + 5\,P_d + C_{u,d} && \forall~d \in D
\end{align*}

\pagebreak
\part{Solution}
Solving this MILP model in Gurobi returns the following solution.
This would cost WonderMarket \$12576018.00 each year (\$241846.50 per week).

\section{Store Assignments}
\begin{tabular}{l  r  r  r r r r r}
    Store & DC0 & DC1 & DC2 & DC3 & DC4 & DC5 & DC6 \\
    S0 &  &  &  & 100.0\% &  &  &  \\
    S1 &  &  &  & 100.0\% &  &  &  \\
    S2 &  &  &  & 100.0\% &  &  &  \\
    S3 &  &  &  &  &  & 100.0\% &  \\
    S4 &  &  &  & 100.0\% &  &  &  \\
    S5 &  & 100.0\% &  &  &  &  &  \\
    S6 &  & 100.0\% &  &  &  &  &  \\
    S7 &  &  & 100.0\% &  &  &  &  \\
    S8 &  &  & 100.0\% &  &  &  &  \\
    S9 &  &  &  & 100.0\% &  &  &  \\
\end{tabular} 

\section{Distribution Centres}
WonderMarket should close DC0 and build DC3 and DC5. DC1 and DC2 
should remain open.

\section{Labour}
WonderMarket should hire the following teams for the whole year. \\[1em]
\begin{tabular}{l  r  r }
    DC & Part-time & Full-time \\
    DC1 & 0 &2 \\ 
    DC2  & 0 &2 \\ 
    DC3 &  0& 8 \\
    DC5 &  0&2 \\ 
\end{tabular} \\[1em]
Additionally, they should hire the following casual staff for certain 
surges.\\[1em]
\begin{tabular}{l  l  r }
    Surge & DC & Casual \\
    Surge 0 &DC5& 11  \\ 
    Surge 1&DC2 & 20 \\ 
    Surge 3&DC2 & 20 \\
    Surge 4&DC1 & 42 \\ 
    Surge 4&DC3 & 2 \\ 
\end{tabular} 
\end{document}