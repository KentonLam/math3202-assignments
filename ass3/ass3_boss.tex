\documentclass[11pt,a4paper]{article}
\input{../../../../header.tex}
\input{../../../../normal.tex}

\usepackage{booktabs}

\usepackage{lmodern}
\renewcommand*\familydefault{\sfdefault} %% Only if the base font of the document is to be sans serif
\usepackage[T1]{fontenc}

\author{Kenton Lam}
\date{{MATH3202 Assignment 3 \\ Due 27/05/2019 1:00 pm}}
\title{WonderMarket \\ Section A -- Internal Report}


\begin{document}
\maketitle
\begin{abstract}
    Our long-time client WonderMarket has approached regarding their foray into the 
    refridgerator space. Competition in the area is fierce and they need us 
    to optimise their fridge logistics.
    Given information about their costs and requirements, we created and
    optimmised a dynamic programming model using Python.
    This report describes the model and its solution.
\end{abstract}

\part{Model Definition}
WonderMarket wishes to start selling fridges. For the time being, they are focusing 
on 3 bespoke fridge options---named ``Alaska'', ``Elsa'' and ``Lumi''.
Over a 4 week trial period, they need to decide how many fridges of each type 
to purchase each week.

After communication 9, we developed a detailed model to mathematically represent
their objectives. This included considering each possible demand scenario to maximise 
expected profit.

\section{Sets}
Let the 3 types of fridges be numbered as $0$ is Alaska, $1$ is Elsa and $2$ is Lumi.
Denote this set as $F$.

\section{Data}
The following data has been provided to us by WonderMarket. 

\begin{tabular}{l l}
    $\mathrm{Profit}_{f}$ & profit made by selling one of fridge $f$. \\
    $\mathrm{Expected}_{f,n}$& expected number of fridge $f$ sold if $n$ are displayed (comm 1). \\
        & ($ 0 \le n \le 4$). \\ 
    $\mathrm{Demands}_{f, n}$ & probability that $n$ units of fridge $f$ will be sold (comm 2).\\
        & ($ 1 \le n \le 6$). \\ 
    $\mathrm{StoreCost}$ & cost of storing one fridge for one week (comm 2). \\
    $\mathrm{FridgesPerTruck}$ & maximum fridges transported by one truck (comm 3). \\ 
    $\mathrm{TruckCost}$ & cost of one truck (comm 3). \\ 
    $\mathrm{MaxTrucks}$ & maximum number of trucks per week (comm 3). \\ 
    $\mathrm{MaxStore}$ & maximum number of each fridge type handled per week (comm 3). \\
\end{tabular}

\section{Stages}
Our stage will be the number of weeks elapsed since the trial began. That is, 
\begin{align*}
    t = 0, 1, 2, 3, 4.
\end{align*}
Because this is only a 4 week trial, we stop once 4 weeks have elapsed.

\section{State}
This is the state carried from one week to the next. Specifically, this is 
number of each type of fridge they have in storage. Denote this as a 3-dimensional 
vector $\vec s$, where each component is
\begin{align*}
    s_f &= \text{number of fridge $f$ stored.} 
\end{align*}

\section{Actions}
At each time step, WonderMarket can order up to 14 fridges, distributed in any way 
among the 3 types. These are the actions they can take. Thus, the action space is given by the set 
\begin{align*}
    \mathrm{Actions} &= \left\{ (i, j, k)  ~|~ i, j, k \in \mathbb Z_{\ge 0}, i+j+k \le \mathrm{FridgesPerTruck}\times \mathrm{MaxTrucks}\right\}.
\end{align*}
Let $\vec a \in \mathrm{Actions}$ denote one specific action.


\section{More Notation}
For a vector $\vec v$, let $v_i$ be the $i$-th component of the vector.
We write the element-wise minimum of two $n$-dimensional vectors as 
\begin{align*}
     \mathrm{emin}( \vec v, \vec u ) &= \left( \min \left\{ v_1, u_1 \right\}, \min \left\{ v_2, u_2 \right\}, \ldots, \min \left\{ v_n, u_n \right\} \right)
\end{align*}
Note that $\mathrm{Profit}$ is a 3-dimensional vector. 

In our value function, we need to consider all possible permutations of demands 
of each fridge type. We write the set of demand permutations as  
\begin{align*}
    \mathrm{DemandPerms} &= \left\{ 1, 2, \ldots, 6\right\}^3
\end{align*}
and denote the probability of each permutation $\vec d \in \mathrm{DemandPerms}$ 
as 
\begin{align*}
    \mathrm{Prob}_\vec d &= \prod_{f \in F} \mathrm{Demands}_{f, d_f}.
\end{align*}

\section{Value Function}
The value function represents the maximum expected profit given a starting time 
and state. It can be expressed as $V_4(\vec s) = 0$, then for $t < 4$,
\begin{align*}
    V_{t}(\vec s) = 
    \max_{ \substack{\vec a \,\in\, \mathrm{Actions} \\ \max \,(\vec s + \vec a)\, \le\, \mathrm{MaxStore} } }\Bigg\{& 
        -\mathrm{StoreCost}\times\sum_{f \in F} (s_f + a_f) \\ 
        &-\mathrm{TruckCost}\times \ceil*{\frac{\sum_{f\in F} a_f}{\mathrm{FridgesPerTruck}}} \\ 
        & + \sum_{\vec d \in \mathrm{DemandPerms}} \mathrm{Prob}_{\vec d} \times \Big( \mathrm{Profit} \cdot \mathrm{emin} ( \vec d, \vec s + \vec a)   \\ 
        & \qquad + V_{t+1} (\vec s + \vec a - \mathrm{emin} (\vec d, \vec s + \vec a) )\Big) \,\,\Bigg\}
\end{align*}

\subsection{Derived Data}
To simplify later calculations, we derived some data from the data given above.\\[0.8em]    
\begin{tabular}{l l}
    $\mathrm{SurgeMultiplier}_{u,s}$ & surge demand at store $s$ during surge $u$,  divided by regular \\ 
    &  demand at store $s$. \\ 
    &$\mathrm{SurgeMultiplier}_{u,s} = \mathrm{SurgeDemand}_{u,s} / \mathrm{Demand}_s $ \\
    $\mathrm{NormalWeeks}$ & number of weeks in the year which have no surge scenario. \\ 
    & $\mathrm{NormalWeeks} = 52-\sum_{u \in U} \mathrm{SurgeWeeks}_u $
\end{tabular}

\section{Variables}
The following variables were used in the Gurobi model.\\[0.8em]
\begin{tabular}{l l}
    $B_{d}$ & binary variables for whether DC $d$ is active. \\ 
    & (1 means if $d$ is new then $d$ is built; if $d$ 
    already exists then $d$ is not closed). \\
    $P_d$ & integer number of part-time teams employed at DC $d$ year-round. \\ 
    $F_d$ & integer number of full-time teams employed at DC $d$ year-round. \\ 
    $C_{u,d}$ & integer number of casual employeed employed at DC $d$  during surge $u$. \\
    & (each casual employee is only employed for the duration of the surge). \\ 
    $A_{d,s}$ & binary variable for whether DC $d$ delivers to store $s$. \\  
    $X_{d,s}$ & integer truckloads to be sent from DC $d$ to store $s$ during normal demand. \\
    $Y_{d,s,u}$ & integer truckloads to be sent from DC $d$ to store $s$ during 
    surge scenario $u$.  \\ 
    & ($X$ and $Y$ will be 0 or exactly match demand, so will be integers.)
\end{tabular}
\\[0.8em]

\section{Objective}
This calculates the total yearly cost, considering transport and labour costs.
The objective is to \textit{minimise} the following function.
\begin{align*}
    \text{(normal transport cost)} \qquad&\quad \sum_{s \in S} \sum_{d \in D} \mathrm{NormalWeeks} \cdot \mathrm{Cost}_{d, s}  X_{d,s} \\ 
    \text{(surge transport costs)} \qquad &+ \sum_{u \in U} \sum_{s \in S} \sum_{d \in D} \mathrm{SurgeWeeks}_u \mathrm{Cost}_{d, s} Y_{d,s,u} \\ 
    \text{(full/part-time labour costs)} \qquad &+ 52 \sum_{d \in D} \mathrm{PTCost}\, P_d + 52 \sum_{d \in D}\mathrm{FTCost} \,F_d \\
    \text{(casual labour costs)} \qquad &+\sum_{u \in U} \sum_{d \in D} \mathrm{CasualCost}\,\mathrm{SurgeWeeks}_u C_{u,d}
\end{align*}

\section{Constraints}
First, we have the constraint that all variables are non-negative and certain variables 
are integers or binary. For all $d \in D,~s \in S, ~u \in U$, we have
\begin{align*}
    B_d, P_d, F_d, C_{u,d}, A_{d,s}, X_{d,s}, Y_{d,s,u}  &\ge 0  \\ 
    B_d, A_{d,s} &\in  \left\{ 0, 1\right\} \\ 
    P_d, F_d, C_{u,d} &\in \mathbb Z
\end{align*}
We link $X$ and $Y$ variables via $A$ and the known demand at each store. This ensures 
each store receives all its supplies from one DC.
\begin{align*}
    X_{d,s} &= \mathrm{Demand}_s A_{d,s} && \forall~d \in D, ~s \in S
\end{align*}
We link $X$ and $Y$ via the surge multipliers data. This ensures for each store, 
it receives deliveries from the same DCs in each scenario as during normal demand. 
\begin{align*}
    Y_{d,s,u} &= \mathrm{SurgeMultiplier}_{u,s} X_{d,s} && \forall~d \in D,~s \in S,~u \in U
\end{align*}
We ensure the solution is valid for normal demand considering 
truckloads and capacities. Note that the northside capacity limit has been removed.
\begin{align*}
    \sum_{d \in D} X_{d,s} &\ge \mathrm{Demand}_{s} &&\forall~s \in S \\ 
    \sum_{s \in S} X_{d,s} &\le \mathrm{Capacity}_d &&\forall~d \in D 
\end{align*}
Ensuring at most 2 new DCs are built and there are 4 DCs in total (3 or 2 old).
\begin{align*}
    \sum_{d \in \mathrm{NewDCs}} B_d &\le 2 \\ 
    \sum_{d \in D} B_d &= 4 
\end{align*}
Ensuring enough teams are employed to meet normal demand at each DC.
\begin{align*}
    \sum_{s \in S} X_{d,s} &\le \mathrm{FTCapacity}\,F_d + \mathrm{PTCapacity}\,P_d && \forall~d \in D
\end{align*}
We ensure our assignments can scale up to each surge scenario while remaining 
feasible with the given constraints. 
For each $u \in U$,
\begin{align*}
    \sum_{d \in D} Y_{d,s,u} &\ge \mathrm{SurgeDemand}_{u,s} &&\forall~s \in S \\ 
    \sum_{s \in S} Y_{d,s,u} &\le \mathrm{Capacity}_d &&\forall~d \in D \\ 
    \sum_{s \in S} Y_{d,s,u} &\le \mathrm{FTCapacity}\,F_d + \mathrm{PTCapacity}\,P_d + C_{u,d} && \forall~d \in D
\end{align*}

\pagebreak
\part{Solution}
Solving this MILP model in Gurobi returns the following solution.
This would cost WonderMarket \$12576018.00 each year (\$241846.50 per week).

\section{Store Assignments}
\begin{tabular}{l  r  r  r r r r r}
    Store & DC0 & DC1 & DC2 & DC3 & DC4 & DC5 & DC6 \\
    S0 &  &  &  & 100.0\% &  &  &  \\
    S1 &  &  &  & 100.0\% &  &  &  \\
    S2 &  &  &  & 100.0\% &  &  &  \\
    S3 &  &  &  &  &  & 100.0\% &  \\
    S4 &  &  &  & 100.0\% &  &  &  \\
    S5 &  & 100.0\% &  &  &  &  &  \\
    S6 &  & 100.0\% &  &  &  &  &  \\
    S7 &  &  & 100.0\% &  &  &  &  \\
    S8 &  &  & 100.0\% &  &  &  &  \\
    S9 &  &  &  & 100.0\% &  &  &  \\
\end{tabular} 

\section{Distribution Centres}
WonderMarket should close DC0 and build DC3 and DC5. DC1 and DC2 
should remain open.

\section{Labour}
WonderMarket should hire the following teams for the whole year. \\[1em]
\begin{tabular}{l  r  r }
    DC & Part-time & Full-time \\
    DC1 & 0 &2 \\ 
    DC2  & 0 &2 \\ 
    DC3 &  0& 8 \\
    DC5 &  0&2 \\ 
\end{tabular} \\[1em]
Additionally, they should hire the following casual staff for certain 
surges.\\[1em]
\begin{tabular}{l  l  r }
    Surge & DC & Casual \\
    Surge 0 &DC5& 11  \\ 
    Surge 1&DC2 & 20 \\ 
    Surge 3&DC2 & 20 \\
    Surge 4&DC1 & 42 \\ 
    Surge 4&DC3 & 2 \\ 
\end{tabular} 
\end{document}