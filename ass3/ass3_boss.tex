\documentclass[11pt,a4paper]{article}
\input{../../../../header.tex}
\input{../../../../normal.tex}

\usepackage{booktabs}

\usepackage[T1]{fontenc}

\usepackage{lmodern}
% \usepackage{cmbright}
\renewcommand*\familydefault{\sfdefault} %% Only if the base font of the document is to be sans serif

\author{Kenton Lam}
\date{{MATH3202 Assignment 3 \\ Due 27/05/2019 1:00 pm}}
\title{WonderMarket \\ Section A -- Internal Report}


\begin{document}
\maketitle
\begin{abstract}
    Our long-time client WonderMarket has approached regarding their foray into the 
    refridgerator space. Competition in the area is fierce and they need us 
    to optimise their fridge logistics.
    Given information about their costs and requirements, we created and
    optimmised a dynamic programming model using Python.
    This report describes the model and its solution.
\end{abstract}

\part{Model Definition}
WonderMarket wishes to start selling fridges. For the time being, they are focusing 
on 3 bespoke fridge options---named ``Alaska'', ``Elsa'' and ``Lumi''.

\section{Data}
The following data has been provided to us by WonderMarket. 

\begin{tabular}{l l}
    $F$ & the set of fridge types. \\
    $\mathrm{Profit}_{f}$ & profit made by selling one of fridge $f$. \\
    $\mathrm{Expected}_{f,n}$& expected number of fridge $f$ sold if $n$ are displayed (comm 1 only). \\
        & ($ 0 \le n \le 4$). \\ 
    $\mathrm{DemandProbs}_{f, n}$ & probability that $n$ units of fridge $f$ will be sold (comm 2).\\
        & ($ 1 \le n \le 6$). \\ 
    $\mathrm{StoreCost}$ & cost of storing one fridge for one week (comm 2). \\
    $\mathrm{FridgesPerTruck}$ & maximum fridges transported by one truck (comm 3). \\ 
    $\mathrm{TruckCost}$ & cost of one truck (comm 3). \\ 
    $\mathrm{MaxTrucks}$ & maximum number of trucks per week (comm 3). \\ 
    $\mathrm{MaxStore}$ & maximum number of each fridge type handled per week (comm 3). \\
\end{tabular}

\section{Communication 1}
WonderMarket wishes to display fridges. They can display up to 8 
fridges and the amount of each fridge bought depends on how many 
fridges of that type are displayed.

Here, the stages are the fridge type and the state is the number of 
fridges remaining. The action describes the number of the current 
fridge to display. We write 
\begin{align*}
    f &= \text{current fridge} \\ 
    r &= \text{remaining fridges} \\ 
    A &= \left\{ 0, \ldots, 4\right\} = \text{all possible number of fridges display} 
\end{align*}

Mathematically, we can write the value function as 
\begin{align*}
    V_f(r) &= \max_{a \in A ~|~a\le r} 
    \left\{ \mathrm{Profit}_f \times \mathrm{Expected}_{f,a} + V_{f+1}(r-a)\right\}
\end{align*}
with the base cases of 
\begin{align*}
    V_3(r) = 0, \qquad V_f(0) = 0 \qquad \forall r, f.
\end{align*}
To optimise for WonderMarket, we computed $V_0(8)$ which will iterate through 
the stage of each fridge (0 to 2) and with 8 available fridges. This resulted in 
a maximum profit of \$1863.20 which is achieved by displaying 
2 Alaska fridges, 3 Elsa fridges and 3 Lumi fridges.

Note that we assumed at most 4 of a fridge type would be displayed. This is reasonable 
as any fridges more than 4 do not increase expected sales. Therefore, it will 
always be better to display some fridges of another type than more than 4 fridges 
of one type.

\section{Communication 2}
In this communication, the focus moved from direct-to-customer sales to delivery.
WonderMarket is running a 4 week trial of their refridgerator branch and 
needs to order fridges to ship to their customers. There are costs 
and constraints associated with storing fridges during this process. 

In addition, 
they cannot know the exact demand ahead of time so require us to optimise for 
maximum expected profit. 

We can consider each fridge independently of the others, so we write a value function 
for one specific fridge then combine them later. For each fridge, 
the stage is number of weeks elapsed, state is number of fridges stored at 
the start of that week and actions represent the number of fridges bought. We also 
have a set of possible demands.

We write 
\begin{align*}
    t &= \text{weeks elapsed} \\ 
    s &= \text{fridges of type $f$ stored at start of week} \\ 
    A &= \left\{ 0, \ldots, 6\right\} = \text{all possible number of fridges to order} \\ 
    D &= \left\{ 1, \ldots, 6\right\} = \text{all possible demands} 
\end{align*}

The value function for a particular fridge can be written as 
\begin{align*}
    V_{f,t}(s) &= \max_{a \in A} 
    \Big\{ -\mathrm{StoreCost}\times(s+a) \\ 
        & \quad + \sum_{d \in D} \left(\mathrm{DemandProbs}_{f,d} \times \left[\mathrm{Profit}_f\times\min(d, s+a)+  V_{f,t+1}(s+a-\min(n, s+a)) \right]\right)
        \Big\}
\end{align*}
with base case $V_{f,4}(s) = 0$. This functions was memoized using 
Python's lru\_cache decorator. We then consider all fridges using 
\begin{align*}
    V_t(a, e, l) &= V_{0,t}(a) + V_{1,t}(e) + V_{2,t}(l).
\end{align*}
Computing $V_0(0,0,0)$ returns \$5282.92 as maximum expected profit which is obtained by 
ordering 4, 5, 5 of Alaksa, Elsa and Lumi fridges respectively in the first week. Because this is 
a stochastic dynamic programming problem, the complete list of optimal actions
cannot be known in advance. You can use the attached stochastic\_explorer.py 
to explore the optimal actions given certain demands in each week.

Note that the summation computes an expected value over all demands.
$\min(n, s+a)$ ensures that the number of fridges sold is limited by the smaller 
of $n$ or $s+a$, the demand and number of fridges available respectively.

We assumed at most 6 of a fridge type can be ordered per week. This is reasonable 
because at most, 6 fridges are sold per week. If 7 fridges are ordered, one would always 
be left over to next week. However, it is cheaper to only order 6 then buy one 
next week if required. Also, because WonderMarket is only just beginning their 
fridge trial, we assume they have no fridge stock stored.

\section{Communication 3}
In addition to the requirements of communication 2, WonderMarket 
needs to consider the logistics of transporting fridges to their warehouse. 
At most, 7 fridges (possibly of mixed type) fit on a truck and at most 2 trucks 
can be ordered per week. Each truck costs a flat fee of \$150. In 
addition, at most 8 fridges of each type can be in the warehouse at a time. 
This results in a maximum of 14 fridges ordered per week.

Again, the stage is number of weeks elapsed. Our state will now be a 
vector of fridges of each type currently stored. 
Similarly, the action is number of each fridge bought in the current week
and the set of demands now considers all 3 fridge types.

Let $\sum \vec x$ be the sum of elements of the vector $\vec x$. 
We write 
\begin{align*}
    t &= \text{weeks elapsed} \\ 
    \vec s &= \text{fridges stored at start of week} \\ 
    & \quad (\text{given}~\vec s, s_f = \text{fridges of type $f$ stored}) \\ 
    A &= \left\{ \vec a \in \left\{ 0, \ldots, 14\right\}^3 ~|~ \sum \vec a \le 14 \right\}  = \text{all possible permutations of fridges to order} \\ 
    & \quad (\text{given}~\vec a \in A, a_f \text{ is number of fridge $f$ ordered}) \\
    D &= \left\{ 1, \ldots, 6\right\}^3 = \text{all possible demands}  \\ 
    & \quad (\text{given}~\vec d \in D, d_f \text{ is demand of fridge $f$}) 
\end{align*}
To keep the notation neat, we define the 
element-wise minimum of two $n$-dimensional vectors as 
\begin{align*}
     \mathrm{emin}( \vec v, \vec u ) &:= \left( \min \left\{ v_1, u_1 \right\}, \min \left\{ v_2, u_2 \right\}, \ldots, \min \left\{ v_n, u_n \right\} \right).
\end{align*}
Also note that $\mathrm{Profit}$ can be treated as a 3-dimensional vector. 
Then, we write the value function as 
\begin{align*}
    V_{t}(\vec s) = 
    \max_{ \substack{\vec a \,\in\, A \\ \max \,(\vec s + \vec a)\, \le\, \mathrm{MaxStore} } }\Bigg\{& 
        -\mathrm{StoreCost}\times\sum(\vec s + \vec a) \\ 
        &-\mathrm{TruckCost}\times \ceil*{\frac{\sum \vec a}{\mathrm{FridgesPerTruck}}} \\ 
        & + \sum_{\vec d \in D} \Big[\Big(\prod_{f \in F}\mathrm{DemandProbs}_{f,d_f} \Big)\times \Big( \mathrm{Profit} \cdot \mathrm{emin} ( \vec d, \vec s + \vec a)   \\ 
        & \qquad + V_{t+1} \big(\vec s + \vec a - \mathrm{emin} (\vec d, \vec s + \vec a)\big)\,\Big) \Big]\,\,\Bigg\}
\end{align*}

\section{More Notation}
For a vector $\vec v$, let $v_i$ be the $i$-th component of the vector.
We write the element-wise minimum of two $n$-dimensional vectors as 
\begin{align*}
     \mathrm{emin}( \vec v, \vec u ) &= \left( \min \left\{ v_1, u_1 \right\}, \min \left\{ v_2, u_2 \right\}, \ldots, \min \left\{ v_n, u_n \right\} \right)
\end{align*}
Note that $\mathrm{Profit}$ is a 3-dimensional vector. 

In our value function, we need to consider all possible permutations of demands 
of each fridge type. We write the set of demand permutations as  
\begin{align*}
    \mathrm{DemandPerms} &= \left\{ 1, 2, \ldots, 6\right\}^3
\end{align*}
and denote the probability of each permutation $\vec d \in \mathrm{DemandPerms}$ 
as 
\begin{align*}
    \mathrm{Prob}_\vec d &= \prod_{f \in F} \mathrm{DemandProbs}_{f, d_f}.
\end{align*}

\section{Value Function}
The value function represents the maximum expected profit given a starting time 
and state. It can be expressed as $V_4(\vec s) = 0$, then for $t < 4$,
\begin{align*}
    V_{t}(\vec s) = 
    \max_{ \substack{\vec a \,\in\, \mathrm{Actions} \\ \max \,(\vec s + \vec a)\, \le\, \mathrm{MaxStore} } }\Bigg\{& 
        -\mathrm{StoreCost}\times\sum_{f \in F} (s_f + a_f) \\ 
        &-\mathrm{TruckCost}\times \ceil*{\frac{\sum_{f\in F} a_f}{\mathrm{FridgesPerTruck}}} \\ 
        & + \sum_{\vec d \in \mathrm{DemandPerms}} \mathrm{Prob}_{\vec d} \times \Big( \mathrm{Profit} \cdot \mathrm{emin} ( \vec d, \vec s + \vec a)   \\ 
        & \qquad + V_{t+1} (\vec s + \vec a - \mathrm{emin} (\vec d, \vec s + \vec a) )\Big) \,\,\Bigg\}
\end{align*}

\subsection{Derived Data}
To simplify later calculations, we derived some data from the data given above.\\[0.8em]    
\begin{tabular}{l l}
    $\mathrm{SurgeMultiplier}_{u,s}$ & surge demand at store $s$ during surge $u$,  divided by regular \\ 
    &  demand at store $s$. \\ 
    &$\mathrm{SurgeMultiplier}_{u,s} = \mathrm{SurgeDemand}_{u,s} / \mathrm{Demand}_s $ \\
    $\mathrm{NormalWeeks}$ & number of weeks in the year which have no surge scenario. \\ 
    & $\mathrm{NormalWeeks} = 52-\sum_{u \in U} \mathrm{SurgeWeeks}_u $
\end{tabular}

\section{Variables}
The following variables were used in the Gurobi model.\\[0.8em]
\begin{tabular}{l l}
    $B_{d}$ & binary variables for whether DC $d$ is active. \\ 
    & (1 means if $d$ is new then $d$ is built; if $d$ 
    already exists then $d$ is not closed). \\
    $P_d$ & integer number of part-time teams employed at DC $d$ year-round. \\ 
    $F_d$ & integer number of full-time teams employed at DC $d$ year-round. \\ 
    $C_{u,d}$ & integer number of casual employeed employed at DC $d$  during surge $u$. \\
    & (each casual employee is only employed for the duration of the surge). \\ 
    $A_{d,s}$ & binary variable for whether DC $d$ delivers to store $s$. \\  
    $X_{d,s}$ & integer truckloads to be sent from DC $d$ to store $s$ during normal demand. \\
    $Y_{d,s,u}$ & integer truckloads to be sent from DC $d$ to store $s$ during 
    surge scenario $u$.  \\ 
    & ($X$ and $Y$ will be 0 or exactly match demand, so will be integers.)
\end{tabular}
\\[0.8em]

\section{Objective}
This calculates the total yearly cost, considering transport and labour costs.
The objective is to \textit{minimise} the following function.
\begin{align*}
    \text{(normal transport cost)} \qquad&\quad \sum_{s \in S} \sum_{d \in D} \mathrm{NormalWeeks} \cdot \mathrm{Cost}_{d, s}  X_{d,s} \\ 
    \text{(surge transport costs)} \qquad &+ \sum_{u \in U} \sum_{s \in S} \sum_{d \in D} \mathrm{SurgeWeeks}_u \mathrm{Cost}_{d, s} Y_{d,s,u} \\ 
    \text{(full/part-time labour costs)} \qquad &+ 52 \sum_{d \in D} \mathrm{PTCost}\, P_d + 52 \sum_{d \in D}\mathrm{FTCost} \,F_d \\
    \text{(casual labour costs)} \qquad &+\sum_{u \in U} \sum_{d \in D} \mathrm{CasualCost}\,\mathrm{SurgeWeeks}_u C_{u,d}
\end{align*}

\section{Constraints}
First, we have the constraint that all variables are non-negative and certain variables 
are integers or binary. For all $d \in D,~s \in S, ~u \in U$, we have
\begin{align*}
    B_d, P_d, F_d, C_{u,d}, A_{d,s}, X_{d,s}, Y_{d,s,u}  &\ge 0  \\ 
    B_d, A_{d,s} &\in  \left\{ 0, 1\right\} \\ 
    P_d, F_d, C_{u,d} &\in \mathbb Z
\end{align*}
We link $X$ and $Y$ variables via $A$ and the known demand at each store. This ensures 
each store receives all its supplies from one DC.
\begin{align*}
    X_{d,s} &= \mathrm{Demand}_s A_{d,s} && \forall~d \in D, ~s \in S
\end{align*}
We link $X$ and $Y$ via the surge multipliers data. This ensures for each store, 
it receives deliveries from the same DCs in each scenario as during normal demand. 
\begin{align*}
    Y_{d,s,u} &= \mathrm{SurgeMultiplier}_{u,s} X_{d,s} && \forall~d \in D,~s \in S,~u \in U
\end{align*}
We ensure the solution is valid for normal demand considering 
truckloads and capacities. Note that the northside capacity limit has been removed.
\begin{align*}
    \sum_{d \in D} X_{d,s} &\ge \mathrm{Demand}_{s} &&\forall~s \in S \\ 
    \sum_{s \in S} X_{d,s} &\le \mathrm{Capacity}_d &&\forall~d \in D 
\end{align*}
Ensuring at most 2 new DCs are built and there are 4 DCs in total (3 or 2 old).
\begin{align*}
    \sum_{d \in \mathrm{NewDCs}} B_d &\le 2 \\ 
    \sum_{d \in D} B_d &= 4 
\end{align*}
Ensuring enough teams are employed to meet normal demand at each DC.
\begin{align*}
    \sum_{s \in S} X_{d,s} &\le \mathrm{FTCapacity}\,F_d + \mathrm{PTCapacity}\,P_d && \forall~d \in D
\end{align*}
We ensure our assignments can scale up to each surge scenario while remaining 
feasible with the given constraints. 
For each $u \in U$,
\begin{align*}
    \sum_{d \in D} Y_{d,s,u} &\ge \mathrm{SurgeDemand}_{u,s} &&\forall~s \in S \\ 
    \sum_{s \in S} Y_{d,s,u} &\le \mathrm{Capacity}_d &&\forall~d \in D \\ 
    \sum_{s \in S} Y_{d,s,u} &\le \mathrm{FTCapacity}\,F_d + \mathrm{PTCapacity}\,P_d + C_{u,d} && \forall~d \in D
\end{align*}

\pagebreak
\part{Solution}
Solving this MILP model in Gurobi returns the following solution.
This would cost WonderMarket \$12576018.00 each year (\$241846.50 per week).

\section{Store Assignments}
\begin{tabular}{l  r  r  r r r r r}
    Store & DC0 & DC1 & DC2 & DC3 & DC4 & DC5 & DC6 \\
    S0 &  &  &  & 100.0\% &  &  &  \\
    S1 &  &  &  & 100.0\% &  &  &  \\
    S2 &  &  &  & 100.0\% &  &  &  \\
    S3 &  &  &  &  &  & 100.0\% &  \\
    S4 &  &  &  & 100.0\% &  &  &  \\
    S5 &  & 100.0\% &  &  &  &  &  \\
    S6 &  & 100.0\% &  &  &  &  &  \\
    S7 &  &  & 100.0\% &  &  &  &  \\
    S8 &  &  & 100.0\% &  &  &  &  \\
    S9 &  &  &  & 100.0\% &  &  &  \\
\end{tabular} 

\section{Distribution Centres}
WonderMarket should close DC0 and build DC3 and DC5. DC1 and DC2 
should remain open.

\section{Labour}
WonderMarket should hire the following teams for the whole year. \\[1em]
\begin{tabular}{l  r  r }
    DC & Part-time & Full-time \\
    DC1 & 0 &2 \\ 
    DC2  & 0 &2 \\ 
    DC3 &  0& 8 \\
    DC5 &  0&2 \\ 
\end{tabular} \\[1em]
Additionally, they should hire the following casual staff for certain 
surges.\\[1em]
\begin{tabular}{l  l  r }
    Surge & DC & Casual \\
    Surge 0 &DC5& 11  \\ 
    Surge 1&DC2 & 20 \\ 
    Surge 3&DC2 & 20 \\
    Surge 4&DC1 & 42 \\ 
    Surge 4&DC3 & 2 \\ 
\end{tabular} 
\end{document}