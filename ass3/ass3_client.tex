\documentclass[11pt,a4paper]{article}
\input{../../../../header.tex}
\input{../../../../normal.tex}

\usepackage{lmodern}
\renewcommand*\familydefault{\sfdefault} %% Only if the base font of the document is to be sans serif
\usepackage[T1]{fontenc}

\author{Kenton Lam}
\date{{MATH3202 Assignment 3 \\ Due 27/05/2019 1:00 pm}}
\title{WonderMarket \\ 
Section B -- Client Report}


\begin{document}
\maketitle
\begin{abstract}
    In this report, we propose a solution to optimise your refridgerator logistics 
    and maximise expected profit. 
\end{abstract}

\part{Solution}
We have considered the customer-facing aspect of fridges put on display 
as well as the logistics of your fridge deliveries. We have arrived at the 
following optimal solutions for each communication.

\section{Communication 1}
In this communication, we know how much profit each fridge can be sold for
and how expected sales vary with the types of fridges placed on display.
Only 8 fridges can be displayed in total.

The maximum expected profit is \$1863.20. This is achieved by displaying 
2 Alaska fridges, 3 Elsa fridges and 3 Lumi fridges. 
In this case, on average you will sell 3.1, 4.2 and 4.1 of Alaska, Elsa and 
Lumi fridges respectively, leading to the profit of 
\begin{align*}
    \text{expected profit} &= 3.1\times \$140+4.2\times\$147+4.1\times \$198 = \$1863.20.
\end{align*}



\section{Communication 2}
In this communication, the focus moved from direct-to-customer sales to delivery
processes. Because customer demand is not known in advance, we need to 
balance ordering and storing them with having enough fridges to meet 
future demand. We will consider a period of 4 weeks.

Using the demand distribution you provided us, the maximum expected profit 
over 4 weeks is \$5282.92 (assuming no fridges to start with). This is done by buying 
4 Alaska fridges, 5 Elsa fridges and 5 Lumi fridges in the first week.

Moreover, in any week, your optimal strategy is 
to have 4, 5 and 5 fridges respectively of each type available.
This means if you have less than this amount stored at the start of a 
week, buy exactly enough to make up this amount. We have verified that 
this is optimal for all feasible demand scenarios.

\subsection{Insights}
In general stochastic (uncertain)
optimisation problems, the solution cannot be expressed so simply (see communication 3).
However, the simplicity of this solution is reasonable because
each fridge does not affect the sales or deliveries of the other fridge types
and demand remains constant through all weeks.

\section{Communication 3}
In this communication, we need to consider the costs of transporting fridges to 
the storage warehouse where they are held before being sent to customers. 
Each truck costs \$150 and has a  limited capacity of 7 fridges. At most 2 trucks 
will be ordered per week and at most 8 fridges of each type can be processed 
at the warehouse. A truck can hold any combination of fridge types so we need 
to consider all fridge types together.

Assuming you currently have no fridges, the maximum expected profit is 
\$4229.46 and you should purchase 4, 5 and 5 respectively in week 1. 

In this case, the solution is more complicated. Because each fridge affects 
other fridges and each fridge sells for a different amount, your buying strategy 
is more nuanced. With this in mind, we have developed an interactive 
solution explorer. You input the amount you sell each week and it recommends 
the optimal buying strategy for the next week. Please see the appendix.

\subsection{Insights}
In all cases, you should never exceed 5 Alaska fridges, 6 Elsa or 6 Lumi fridges.
These are the maximum which can be sold per week. Buying any more than this amount 
will require excess stock to be stored to the next week, raising costs. 

Although the optimal strategy is varied, the there is a threshold below which you 
should order more fridges. Because trucks cost money, you should order 
fridges in batches of 7 or 14 unless it violates the above rule.

To try visualise this multi-dimensional data, we have included a 3D plot below. 
This is at the beginning of the first week and the axes are number of each fridge type stored. Red is Alaska, green is Elsa 
and blue is Lumi. The colour of each dot represents how many fridges 
should be ordered on a scale with green being 0 and red being 14.
\begin{center}
\includegraphics[width=20em]{cube.png}
\end{center}
Here, you can see that there are approximately diagonal thresholds where 
transitions between truckloads ordered occur. However, these aren't exact; 
for example even if you have 5 Alaska fridges but no other fridges, you should 
still order 2 truckloads.

Broadly, the optimal actions are grouped into 3 bunches---0, 1 and 2 truckloads. 
Most of the cases in each group will order the maximum possible but 
occasionally, you will want to order less due to the first paragraph in this section.



We determined \textbf{opening DC3} would be best. This 
 would reduce the weekly cost to 
\$160486 which is \$45202 cheaper than communication 5. 
The store assignments are below. \\[1 em]
\begin{tabular}{l  r  r  r r r r r}
    Store & DC0 & DC1 & DC2 & DC3 & DC4 & DC5 & DC6 \\
    S0 &  &  &  & 100\% &  &  &  \\
    S1 &  &  &  & 100\% &  &  &  \\
    S2 &  &  &  & 100\% &  &  &  \\
    S3 &  & 100\% &  &  &  &  &  \\
    S4 &  &  &  & 100\% &  &  &  \\
    S5 &  & 100\% &  &  &  &  &  \\
    S6 &  & 100\% &  &  &  &  &  \\
    S7 &  &  & 100\% &  &  &  &  \\
    S8 &  &  & 100\% &  &  &  &  \\
    S9 &  &  &  & 100\% &  &  &  \\
\end{tabular}


\section{Communication 7}
Communication 7 builds on communication 6. Now, we can open up to 2 new DCs. 
If we open 2 new DCs, 1 old DC must be closed.
We consider if it would be cheaper
to build 2 of the new DCs and close an existing one.\@  

Note that in communication 6, DC0 is not used at all despite being available. 
Thus, closing it would not affect the weekly cost. Moreover, 
an additional new DC could be opened to further reduce the cost. This 
communication's model recommends \textbf{closing DC0} and 
\textbf{opening DC3 and DC5}. 

This results in a weekly transport cost of \$140776, \$19710 cheaper  than communication 
6, with the following assignments.\\[1em]
\begin{tabular}{l  r  r  r  r r r r}
    Store & DC0 & DC1 & DC2 & DC3 & DC4 & DC5 & DC6 \\
    S0 &  &  &  & 100\% &  &  &  \\
    S1 &  &  &  & 100\% &  &  &  \\
    S2 &  &  &  & 100\% &  &  &  \\
    S3 &  &  &  &  &  & 100\% &  \\
    S4 &  &  &  & 100\% &  &  &  \\
    S5 &  & 100\% &  &  &  &  &  \\
    S6 &  & 100\% &  &  &  &  &  \\
    S7 &  &  & 100\% &  &  &  &  \\
    S8 &  &  & 100\% &  &  &  &  \\
    S9 &  &  &  & 100\% &  &  &  \\
\end{tabular}

\newpage

\section{Communication 8}
Here, we also consider labour costs during regular demand.

Optimising the model results in a cost of \$202276 per week, \$61500 more than 
comm 7. This is done by \textbf{closing DC0} and 
\textbf{opening DC3 and DC5} with the below store and labour assignments. 
Note that the store and DC assignments are the same as 
communication 7 so this does not affect those; the additional \$61500
is exactly the cost of the full-time and part-time teams.
\\[0.8em]
\begin{tabular}{l  r  r  r  r r r r}
    Store & DC0 & DC1 & DC2 & DC3 & DC4 & DC5 & DC6 \\
    S0 &  &  &  & 100\% &  &  &  \\
    S1 &  &  &  & 100\% &  &  &  \\
    S2 &  &  &  & 100\% &  &  &  \\
    S3 &  &  &  &  &  & 100\% &  \\
    S4 &  &  &  & 100\% &  &  &  \\
    S5 &  & 100\% &  &  &  &  &  \\
    S6 &  & 100\% &  &  &  &  &  \\
    S7 &  &  & 100\% &  &  &  &  \\
    S8 &  &  & 100\% &  &  &  &  \\
    S9 &  &  &  & 100\% &  &  &  \\
\end{tabular} \\ [0.8em]
\begin{tabular}{l  r  r }
    DC & Part-time & Full-time \\
    DC1 & 0 &2 \\ 
    DC2  & 3 &0 \\ 
    DC3 &  0& 8 \\
    DC5 &  3&0 \\ 
\end{tabular}

\newpage
\section{Communication 9}
Finally, we also consider the transport and labour costs of 
surge scenarios across the whole year. 

Solving this model results in a cost of \$12576018.00 each year 
(on average, \$241846.50 per week), an increase of \$39570.50 
per week from communication 8.
You should \textbf{close DC0} and 
\textbf{open DC3 and DC5} 
with the below store and labour assignments. 
Again, the store and DC assignments are the same as communications 
7 and 8 but the full/part-time teams have changed.\\[0.8em]
\begin{tabular}{l  r  r  r  r r r r}
    Store & DC0 & DC1 & DC2 & DC3 & DC4 & DC5 & DC6 \\
    S0 &  &  &  & 100\% &  &  &  \\
    S1 &  &  &  & 100\% &  &  &  \\
    S2 &  &  &  & 100\% &  &  &  \\
    S3 &  &  &  &  &  & 100\% &  \\
    S4 &  &  &  & 100\% &  &  &  \\
    S5 &  & 100\% &  &  &  &  &  \\
    S6 &  & 100\% &  &  &  &  &  \\
    S7 &  &  & 100\% &  &  &  &  \\
    S8 &  &  & 100\% &  &  &  &  \\
    S9 &  &  &  & 100\% &  &  &  \\
\end{tabular} \\ [0.8em]
\begin{tabular}{l  r  r }
    DC & Part-time & Full-time \\
    DC1 & 0 &2 \\ 
    DC2  & 0 &2 \\ 
    DC3 &  0& 8 \\
    DC5 &  0&2 \\ 
\end{tabular} \\ [0.8em]
\begin{tabular}{l  l  r }
    Surge & DC & Casual \\
    Surge 0 &DC5& 11  \\ 
    Surge 1&DC2 & 20 \\ 
    Surge 3&DC2 & 20 \\
    Surge 4&DC1 & 42 \\ 
    Surge 4&DC3 & 2 \\ 
\end{tabular} \\[0.8em] (Surge 2 needs no additional casual staff at any DC)

\newpage
\part{Insights}
\section{Cost Breakdown}
Below, you can find a table showing the weekly breakdowns of normal 
demand and surge scenarios. Note that the full-time and part-time 
teams are employed for all 52 weeks each year. \\[1.5em]


\section{General Observations}
While optimising the assignments with the data you provided us, we made 
the following observations.
\begin{itemize}
    \item The optimal solution to communication 9 uses no part-time teams, despite 
    them being used in communication 8. This is because the additional 
    capacity from full-time staff will be used during surge
     scenarios, reducing the need for expensive casual workers. Each casual worker only 
     handles 1 truckload, but costs 65\% as much as a full-time team which handles 
     9 truckloads. Note how 
     surge 2 can be handled by the only full-time staff.
    \item With communication 9, we have a fairly complete picture of WonderMarket's
    logistics needs and costs. Some DC capacities are never fully utilised, 
    but the model accounts for this and only assigns enough staff to handle 
    the required demand. This means if demand increases, you can simply employ more 
    staff (as long as DC capacity is sufficient). However, we would recommend
    re-evaluating the model to optimise for the new situation.
    \item With the construction of DC3 and DC5, your total capacity has increased 
    notably. Even the most demanding surge scenario, surge 4 with 164 truckloads,
    has 55 truckloads of spare capacity across DC1, DC2 and DC5.
    \item With that in mind, DC3 is nearing its capacity. There are at most  
    4 unused truckloads per week. In surge scenario 4, all of its 
    capacity is used. 
\end{itemize}

\section{Marginal Costs}
These are the effects of slight adjustments to the constraints 
you have provided us. These may be useful to improve your management of 
 distribution centres or logistics. 
\begin{itemize}
    \item With the notes on DC capacity above, increasing the DC1, DC2, DC3 or DC5's
    capacity by 1 does not affect the cost as the DC capacity is no longer a 
    limiting factor.
    \item If you can close one more DC to open a new site, you should close 
    DC1 and open DC6. This would reduce your yearly cost by \$853344 (\$16410.46 per week),
    to \$11722674 (\$225436.04 per week).
    \item The solution assigns 5 of the 10 stores to DC3. If DC3 could not be built, 
    your cost would be increased by \$1225564 yearly (\$23568.54 weekly) 
    to \$13801582 yearly (\$265415.04 weekly). Using DC0, DC1, DC2 and DC6 would 
    then be optimal.
\end{itemize}

\end{document}